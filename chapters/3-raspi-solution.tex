\chapter{Raspberry Pi-based Solution}
    \section{Rationale}
        \todo{What will make this a good choice?}
        I have chosen to create a solution using the popular Raspberry Pi computer at its core.
        
        \todo{Advantages
            \begin{itemize}
                \item Full featured computer, runs Fedora in particular (dogfooding)
                \item Fairly cheap
                \item Four USBs onboard
                \item Less worry with soldering
                \item Many people are already familiar with it as it's popular among hobbyist projects
            \end{itemize}
        }
        \todo{Disadvantages !!!}
        \blind[2]
    \section{Specification}
        \todo{What, exactly, will the device consist of?}
        The device shall consist of:
        \begin{itemize}
            \item Raspberry Pi~3 Model~B.  Compatibility with Raspberry Pi~2 Model~B may be possible, but not a primary goal.
            \item A compatible 3.5~inch touchscreen.
            \item A MicroSD card holding the system and several images.
            \item Two male-to-female USB cables.
        \end{itemize}
    
    \section{Software}
        The Fedorator supports both Raspbian, the officially supported Raspberry Pi compatible Linux distribution\cite{raspian}, and the eponymous Fedora operating system.  To faciliate easier testing, it's also possible to start the software on a regular Linux desktop.
        
        The current version of the Fedorator software may be found in the appendix.  It is also made available online on GitHub under the \todo{TODO} license\cite{fedorator-github}, where continued development will take place.  
        
        The software running on the Fedorator is written in Python and uses the Kivy framework to provide a touchscreen interface.  It is is designed to start immediately after boot and stay running indefinitely.
        
        In standby, the screen shows prominently the Fedora logo as well as the name of the device (\ref{pic:fedorator-screenshot-standby.png}).  It prompts the user to insert a flash drive.
        
        \screenshotfigure{fedorator-screenshot-standby.png}
            {The standby screen}
        
        When the user inserts a flash drive, the text changes, stating "Tap to begin".  The size of the flash drive is also displayed at the bottom part of the screen.
        
        Tapping will transition the screen into a menu listing releases available for writing (\ref{pic:fedorator-screenshot-list.png}).  This list can be browsed by a swiping a finger or stylus across the screen.  Tapping the top left corner will always bring the user back to the previous screen, as indicated by the faint "back" text. 
        
        \screenshotfigure{fedorator-screenshot-list.png}
            {Picking a release from the list}
        
        After tapping a desired release, a screen with details is presented (\ref{pic:fedorator-screenshot-details.png}).  It contains a short official description of the release and offers to set the version and architecture, should the user wish for something other than the defaults.
        
        \screenshotfigure{fedorator-screenshot-details.png}
            {Taking a look at the Fedora Workstation release}
        
        If the user is happy with the choice, they can press the sizeable "Flash" button.  This begins the process of writing the image to the flash drive.
        
        \todo{(fedorator) Do I want to change the button from "Flash" to "Write" or such?  Also, Maybe add a warning about this being a destructive action}
        
        \todo{If at any point the flash drive is removed, ...}
        
        \todo{text}
         
    \section{Construction of a Prototype}
        \todo{A prototype will be built.  This should probably include pictures and a short description of the process}
        \blind[3]
    \section{Simple instructions}
        \todo{These are the final instructions that will be the "result" of the work.  Should they be inlined here?}
        \subsection{Bill of Materials}
            \todo{What is needed to get started - include a table and cost}
            \blind[3]
        \subsection{Step-by-step tutorial}
            \todo{The meat: how to build it from grounds up}
            \todo{Will not be present here, rather as an attachment.}
    \section{Conclusions}
        \todo{Words about if it'll be a good fit, after considering the above}
        \todo{Speeds should be included somewhere in here}
        
        \blind[2]
