\chapter{Research and Analysis}
    In order to determine the best way to build the Fedorator device, as well as gather inspiration, I have done research into the problem space.
    
    \section{How Fedora Spreads Publicity}
        \svgfigure{fedora-logo-and-wordmark}{Fedora logo and wordmark \cite{fedora-graphic-standards-manual}}
        The Fedora project has a worldwide Ambassador program.  \textbf{Ambassadors} are people who act as representatives of Fedora.  They have a good understanding of Fedora's principles and aim to spread the message to the public \cite{fedora-ambassadors}.  One of the tasks assigned to Fedora Ambassadors is organizing Fedora participation at events.
        
        In the Czech Republic, Fedora holds stands on events such as DevConf \cite{devconf}, PyCon CZ \cite{pycon-cz}, or Linux Days \cite{linux-days}.  An Ambassador's task at the stand might be to set it up, bring out the merchandise, and then speak with visitors, who may be familiar with Fedora and have questions, or have never heard about it before.
        
        Fedora Ambassadors are not necessarily people of technical expertise.  For example, a Fedora Ambassador might also be involved in the Fedora project as a translator.
        
    \section{Existing solutions}
        Several existing options for distributing USB flash drives were considered.  These options may also offer inspiration for the Fedorator.
        
        \subsection{Preloaded USB flash drives}
            Many services offering bulk preloaded USB flash drives may be found on the Internet.  With a large focus on marketing, in general they offer to preload materials such as PowerPoint presentations and PDF brochures \cite{flashbay-data-preloading}.
            
            Literally, they instruct to send the files by email or upload them.  Hence, it stands a question whether they would be capable of producing bootable USB flash drives.
            
            I went looking to order with the following parameters.
            \begin{itemize}
                \item Capacity of 2GB
                \item 2GB of preloaded data
                \item USB flash drive must be bootable
                \item No branding necessary
                \item 500 units
            \end{itemize}
            
            Most services, rather than show a cost upfront, appear to ask the user to fill in a form and get a personalized quote \cite{memorysuppliers} \cite{flashbay} \cite{premiumusb}.  I have asked for a quote from two services and received no response.  The company MemoTrek has an online quote calculator and offers for my specified parameters the price of 4.19~USD (about 100~CZK) per flash drive \cite{memotrek-clip-n-easy}.  However, they only offer data duplication of up to 1~GB, while most bootable images go over that volume.
            
            For a bit of perspective, the price for 500 replicated DVDs come down to 0.88~USD (about 22~CZK) for one DVD \cite{discmakers-quoter}.  We can see USB flash drives are over four times as expensive.
        
        \subsection{USB Duplicators}
            Rather than using a service and ordering flash drives already preloaded, we can acquire dedicated hardware and do the task ourselves.  There are several products available on the market, called "USB Duplicators", which claim to provide this functionality.  Some examples are provided, with the number of ports and to-date cost in USD and CZK (only informative).
            
            \begin{table}[htbp]
            \centering
            \caption{Examples of USB Duplicators available on the market}
            \label{usb-duplicators}
                \begin{tabular}{ m{15em}  c  r  r  c }
                \toprule
                    \textbf{Name} & \textbf{ports} & \textbf{USD} & \textbf{CZK} & \\
                \toprule
                Altarec USB Copy Tower SA Duplicator
                 & 118 & \$15~749 & 388~225 CZK &  \cite{product-altarec-118} \\
                \hline
                TEAC 1 to 15 USB Drive Duplicator 
                 & 15  & \$1~206  & 29~728 CZK  &  \cite{product-teac-15} \\
                \hline
                VINPOWER Black 1 to 11 USBShark USB Flash Copy Tower Duplicator
                 & 11  & \$890    & 21~668 CZK  &  \cite{product-vinpower-11}        \\
                \hline
                Altarec USB Copy Cruiser Mini Duplicator\textdagger
                 & 10  & \$315    &  7~764 CZK  &  \cite{product-altarec-10}        \\
                \hline
                Intelligent 9 Series (UB910G) - GOLD Factory series 1 - 9 Target USB Flash Memory/Pen Drive/External USB Hard Drive Duplicator
                 & 9   & \$1~260  &  31~056 CZK &  \cite{product-intelligent-9}     \\ 
                \hline
                USB Stick Copystation\textdagger
                 & 7   & Unk.     &        Unk. &  \cite{product-copystation} \\ 
                \hline
                EZ Dupe 2
                 & 3   & \$81     &   1~997 CZK &  \cite{product-ezdupe}       \\
                \hline
                \end{tabular}
            \end{table}
            
            \imagefigure{vinpower-usbshark.jpg}
                {VINPOWER Black 1 to 11 USBShark  \cite{product-vinpower-11}}
            \imagefigure{altarec-copy-cruiser.jpg}
                {Altarec USB Copy Cruiser Mini Duplicator  \cite{product-altarec-10}}
            
            In general, the products can be split into two categories: devices that work stand-alone and devices that require a computer to operate.  Devices from the latter category are marked in the table with a dagger (\textdagger).
            
            Besides their price, the trouble with these products is that they are proprietary.  The hardware cannot be built from scratch and the software cannot be inspected or improved.  Support for creating bootable USB flash drives is not in most cases guaranteed, and indeed people have been burned before \cite{open-security-research-simple-duplicator}.  The devices which are not stand-alone require a computer with specific software and drivers to operate.  Linux support is not claimed, which can usually be interpreted as unavailable, leading to a sad and undesired irony of requiring a computer running an OS that is not Linux in order to produce and distribute USB flash drives containing Linux.
        \subsection{Existing software}
            A number of software exists to serve the purpose of creating live USB flash drives.  The Fedora Project wiki describes several of them and primarily recommends the use of Fedora Media Writer \cite{fedora-how-to-live-usb}.
            
            \subsubsection{Fedora Media Writer}
                \imagefigurelarge{fedora-media-writer-screenshot.png}
                    {Fedora Media Writer running on Linux  \cite{fedora-media-writer-screenshot}}
                
                \textbf{Fedora Media Writer} is a desktop application developed by  Jiřá Eischmann and Martin Bříza.  It covering the entire process from picking a Fedora release and downloading it, to copying to an USB flash drive.  It was overhauled in 2016 to provide a new user friendly interface \cite{fedora-luc-as-primary-downloadable}.
                
                Fedora Media Writer is written in C++ using the Qt framework and supports Linux, Windows, and macOS \cite{fedora-media-writer}.  It's a solid and well tested application, however, clearly suited for desktop usage rather than embedded.  I have decided to create software for the Fedorator from scratch rather than attempt to add a new user interface to Fedora Media Writer.  However, the user interface of Fedora Media Writer did prove an inspiration.
                
            \subsubsection{Other specialized software}
                Another piece of software called \textbf{UNetbootin} allows for writing a wide variety of live distributions onto flash drives.  Its user interface is not as sleek as Fedora Media Writer's, lacking in logos and screenshots.  The Fedora images installed by UNetBootin are known to exhibit issues \cite{fedora-how-to-live-usb} \cite{unix-stack-exchange-error-installing-fedora-24}.
            
        \subsection{Dedicated open source hardware}
            I have attempted a search to discover if somebody has created an open source and open hardware USB copier before.  I found many small scripts for copying USB flash drives \cite{github-usbmk} \cite{github-auc-automaticly-usb-copier} \cite{github-udev_serialcopier}.  None had any sort of hardware design.
            
            The blog post titled "A Simple USB Thumb Drive Duplicator on the Cheap" by Open Security Research \cite{open-security-research-simple-duplicator} offers advice on how to use an USB hub and Linux commands to duplicate to several flash drives at once. 
            
    \section{Rationale for a New Solution}
        There are several advantages to designing a solution from scratch.
        
        \begin{itemize}
            \item It will be a free and open solution under a permissive license, allowing others to contribute improvements.
            \item It allows us to create a customized interface with image options.
            \item It means we can design customized and attractive branding.
            \item The total cost may be lower.
            %\item \todo{More advantages?}
        \end{itemize}
        
        There are also several disadvantages. 
        \begin{itemize}
            \item A person needs to invest time into building the device.
            \item It may not be as compatible, or "battle tested" as commercial solutions\footnote{Although as we have seen, commercial solutions are not always entirely compatible either.}.
            %\item \todo{More disadvantages?}
            \item We may run into other unexpected issues.
        \end{itemize}
        
        Despite available options, as we've seen, they do not offer all of the advantages of possible a new solution.  I will go ahead and design a device called the Fedorator.
