\chapter{Microcontroller-based solution}
    %\todo{If applicable}
    %\todo{After experimentation, I have determined that the Arduino Nano is not viable,...}
    %\todo{photo of the pretty fedorator oled logo at least}
    
    I made an attempt at designing and creating a compact variant of the Fedorator modelled after Concept C  (fig. \ref{pic:concept-c}), dubbed Fedorator Tiny.  I set to work with some components I already had ready.  I began working on a version consisting of an Arduino Nano device, a monochrome 128x64 OLED display, a microSD card reader, and the USB Host Mini board \cite{usb-host-shield}.  For drawing on the display, I used the \texttt{esp8266-oled-ssd1306} library \cite{esp8266-oled-ssd1306}.
    
    The limitations of the Arduino device cropped up quickly, when the code, together with drivers for the display and SD card reader, stopped fitting inside the small 32~KB flash memory.
    
    I reproduced the same work on a NodeMCU device, but had troubles with the SD card library.
    
    I'm now convinced that the ideal core for the Fedorator Tiny would be \textbf{Teensy 3.6}.  With a SD card slot on board as well as a built in USB host capability, it requires no external active components to fulfill all requirements.
    
    There is a library for working with the USB host port in development by the Teensy creator, \texttt{USBHost\_t36} \cite{usbhost_t36}.  In the present, it only supports keyboard and MIDI devices.  The best course of action here would be to add mass storage functoinality to this library.  I have begun the work on this by studying the USB and SCSI protocols.  Code-wise, I performed a small trial in the form of detecting the connection of an USB flash drive as seen on figure \todo{}, but full support would be the subject of another work of a significant scale.
    
    All work was done in C using the PlatformIO Core framework, which provides a unified build system for a large number of development platforms\cite{platformio-core}.
    
    \todo{test in appendix}
    \todo{check citations in this chapter}
    %\todo{"Microcontroller based solution"}
    %\todo{platformio}
    
    \imagefigure{photos/fedorator-tiny-oled.jpg}
        {The Fedora logo on an OLED screen driven by Arduino Nano}
    
    
\begin{comment}

    SO WHAT WILL I BE DOING HERE.
    
    Device dubbed Fedorator Tiny
    I began working with the Arduino Nano because I had it available
    I connected OLED and SD library.
    photo here
    I quickly discovered there is not enough memory.
    
    Briefly considered ESP, but couldn't get the SD module working with it
    
    Decided the Teensy device will be ideal, especially since I discovered it has an USB host.
    
    There is an USB host library, sadly it does not support mass storage yet.
    
    There also exists the USB host board.  It has a mass storage library.  <-- YOU ARE HERE
    
    I made a prototype which copies an image from the SD card to the board.  <-- HOPEFULLY

\end{comment}
