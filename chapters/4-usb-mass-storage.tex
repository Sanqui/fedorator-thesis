\chapter{Interlude: USB Mass Storage}
    \section{}
    In order to create a device behaving as an USB host, I have studied the USB specification.
    
    \todo {Also in order to understand speed, behavior, etc...}
    
    \todo{These are just notes for now.}

\begin{verbatim}
    I will only support 2.0 since that's what most mass storage devices have
    
    * Cable bus
    * Host-scheduled, Token based protocol
    * USB supports up to n peripherials attached simultaneously more but I will only use one
    
    * A topology is prescribed.  Host, hub, and function
    
    * Host: host controller, hw/fw/sw
    
    * Device: Hubs or functions
    
    * Four-wire cable, three data rates
        * High-speed
        * Full-speed
        * Low-speed 
    * Clock
    * VBUS (5V) and GND
    
    * Power: host supplies power.  Bus-powered x self-powered.
    
    * Protocol:
        * Polled bus
        * Host controller initiates
        * Most bus transaction involve transmisison of up to three packets
        * Token packet first
        * Pipes: stream (no defined structure) and message; default control pipe
        * Flow control is offered for stream pipes
        
    * USB offers robustness: error detection and handling
    
    * Attachment goes through a hub, the device receives an address
    
    * Four types of data transfers:
        * Control - configuration and pipe management
        * Bulk - large and bursty, wide latitude and transmission constraints
            * Sequential
            * Reliable
            * Error detection in hardware
            * Variable bandwidth
        * Interrupt - timely and reliable
        * Isochrronous - prenegotiated latency streaming/realtime
    
    * Bandwidth is allocated among pipes; USB devices are required to provide some buffering; of course bandwidth can be spread out.
    
    * USB devices
        * Device classes TODO (Ch 11)
        * All devices support at least one pipe at endpoint zero 9the control pipe)
        
    * Hubs are a thing, but we don't care about them
        * (Check if we are a single hub)
        * Attachment point is a "port"
    
    * Function is an USB device.  
    
    * USB Host interacts with USB devices through the Host Controller.  Its responsibilities are managing the USB devices.
    
    * Blah blah, usb allows for complex topology, but we will only be using host-root hub-device/function.
        
    * Device endpoint
        * Uniquely identifiable portion of an USB device that is the terminus of a communication flow between host and device.
        * Endpoints are independent
        * Endpoint numbers, device-determined direction of data flow
        * <device address, endpoint number, direction> = unique reference
        * Endpoints are simplex with data flow in a single direction.
        * USB Pipe is an association between endpoint on device and software on host
            * Data transfers are requested via I/O Request Packets (IRPs) to a pipe.
            * On STALL or errors for IRPs, IRP is aborted/retried, all outstanding IRPs are retried, and no further IRPs are accepted until software recovers and acknowledges halt/error condition via USBD call
            * Maximum packet size
            * Endpoint can inform that it's busy by responding with NAK.  Not a retry condition.  Is not an error.
            
            * Stream pipes
                * No USB-required structure
                * Unidirectional FIFO
                * Bulk/isochronous/interrupt
            * Message pipes (5.3.2.2) TODO if necessary
        * 5.3.3 frames and microframes
        * transfer types (again)
            * control
            * iso
            * interrupt
            * Bulk transfers
                * Large amounts of data, variable times
                *Access to the USB on a bandwidth-available basis
                * Retry of transfers
                * Guaranteed delivery of data, no guarantee of bandwidth and latency
                * Payload sizes: 8 to 512 (high-speed)
                * Bulk transfers and control registers deliver independently and possibly before each other
            
                
    
    
\end{verbatim}
    \todo{SPEEDS (and average TIMES)}


