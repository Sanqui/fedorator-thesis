\chapter{Interlude: USB Mass Storage}
    USB, standing for Universal Serial Bus, is an industry standard defining the hardware and protocols surrounding connection and communication of electric devices to computers\todo{CITE}.  
    
    The USB 1.0 specification was introduced in 1996 and defined data tranfer rates of 1.5~Mbit/s and 12~Mbit/s\cite{hui1394}. \todo{did fast speed only get introduced with 1.1?}  In 2000, the USB 2.0 specification was released, which a significantly higher maximum data transfer rate of 480~Mbit/s\todo{CITE}.
    
    The USB 3.0 specification, released in 2008, defines a new SuperSpeed transfer mode.  This mode is claimed to support a transfer speed of up to 5~Gbit/s in full-duplex, with 350~MB/s thoughput in current implementations\cite{usb-3-hp}.  SuperSpeed plugs are distinctively identified by blue inserts. \todo{USB 3.1...?}
    
    USB flash drives are data storage devices implementing the USB interface.  Many USB flash drives already implement USB~3.0\cite{wirecutter-best-usb-3-0}, however, all have backwards compatibility as long as they use a traditional USB‑A port.  For our purposes, it's reasonable to assume people will use any old flash drive they have lying around, rather than specifically brand new ones.
    
    In order to create a device behaving as an USB host, I have studied the USB~2.0 specification \todo{name/cite}.  Even if I won't be working with USB at a particularly low level, an overview is essential in order to understand behavior and judge parameters such as speed.
    
    The device I will be designing will only need to support the connection of USB flash drives to a single USB port, so only relevant information is taken in consideration.
    
    \section{USB specification}
        USB is a cable bus providing data exchange between a host device (usually a computer) and several connected peripherals which may be attached and used independently.  USB defines a host-scheduled and token-based protocol for communication with peripherals.
        
        A USB system is prescribed as a tiered star topology involving hosts, hubs, and functions (logical devices).  USB allows for complex asymmetrical topology, but in our situation, only a root hub will be present.
        
        \subsection{USB host}
            A single host is present in any USB system.  The Host Controller, implemented in hardware or software, has many responsibilities.  It initiates connections and manages all communication and bandwidth between software and connected devices.
            
            The USB host also supplies power to connected devices.
        
        \subsection{USB device}
            A USB device is a piece of peripherial connected to the host utilizing the USB protocol.  There is a number of class codes defined in the USB standard\cite{usb-class-codes}.  Common classes include the HID (Human interface device) class encompassing keyboards and mice, the printer class, the video class, and indeed, the mass storage class. 
        
        \subsection{USB hardware}
            USB uses a four-wire cable, encompassing a twisted pair of data wires and as a V\textsubscript{BUS} and GND pair delivering a nominal +5~V power to devices.
            
            The clock is transmitted along with data using a clock encoding scheme.  There are three data rates available:
            \begin{itemize}
                \item Low-speed, defined at 1.5 Mb/s with limited capacity.
                \item Full-speed, defined at 12 Mb/s.
                \item High-speed, defined at 480 Mb/s.
            \end{itemize}
        
        \subsection{USB protocol}
            On the low level, USB functions as a polled bus.  Transactions, as initiated by the Host Controller, usually involve the transmission of up to three packets.  The token packet allows devices to identify whether they're the intended recipient of the packet.
            
            To faciliate communication, the host and device negotiate device endpoints.  A device may be composed of several independent endpoints.  Endpoints are simplex - data flow is defined in a single direction.
            
            A USB pipe is an association between an endpoint on a device and software on the host.  In other words, pipes are what the software handles and uses for data transmission.  Pipes can be configured to behave in \emph{message} or \emph{stream} modes, which either do or do not have a USB-defined structure, respectively.
            
            Pipes define a transfer type and hold several characteristics, such as directionality, the transfer type, the amount of bandwidth allocated, and maximum data payload sizes.  A pipe called the Default Control Pipe is always present which is used for device configuration.
            
            %Data transfers over pipes are requested using I/O Request packets (IRPs).  IRPs will be retried under failure conditions, but three bus errors or a stall will require the host to recover.
            
            USB offers four transfer types for a pipe:
            \begin{itemize}
                \item \textbf{Control Transfer}: Simple request/response communication used to communicate commands and status.
                \item \textbf{Isochronous Transfer}: Contunious communication without guarantee for errors inteded for real-time data.
                \item \textbf{Interrupt Transfer}: Timely and reliable latency-bound communication.
                \item \textbf{Bulk Transfer}: Large and bursty, but non-time sensitive data transmission, utilizing any bandwidth available.
            \end{itemize}
            
            \todo{mass utilize bulk, more on page 52}

    \section{USB mass storage device class}
        The USB mass storage device class, or MSC for short, is a set of protocols allowing an USB device to enable the host to directly access a data storage.  A device implementing this protocol usually appears as an external hard drive.  All sorts of devices implement MSC, including hard drives, optical drives, solid state drives, card readers, or instruments such as cameras and mobile phones\cite{usb-keil-msc}.
        
        A number of subclasses and transport protocols is defined for the MSC.  MSC allows for certain protocols defined outside of USB to be used, such as SCSI or more obscure ones like QIC-157 or SFF-8070i.  The USB defined protocols include the Bulk-Only transport (BBB or BOT) and the obsolete Control/Bulk/Interrupt (CBI) transport, only in use for USB floppy disk drives\cite{usb-mass-storage}\cite{usb-mass-bulk}.
        
        Particularly high performance mass storage devices tend to use the SCSI standard\cite{usb-electronicdesign-usb-uasp-bot}, but because USB Attached SCSI protocol (UASB) was only specified when USB 3.0 was introduced, few devices have adopted it so far\cite{usb-winaero-usb-scsi-uas}.  In particular, as late as 2012, almost all USB flash drives on the market implemented only the BBB protocol\cite{usb-qemu-usb-storage}.

\begin{comment}
    I will only support 2.0 since that's what most mass storage devices have
    
    * Cable bus
    * Host-scheduled, Token based protocol
    * USB supports up to n peripherials attached simultaneously more but I will only use one
    
    * A topology is prescribed.  Host, hub, and function
    
    * Host: host controller, hw/fw/sw
    
    * Device: Hubs or functions
    
    * Four-wire cable, three data rates
        * High-speed
        * Full-speed
        * Low-speed 
    * Clock
    * VBUS (5V) and GND
    
    * Power: host supplies power.  Bus-powered x self-powered.
    
    * Protocol:
        * Polled bus
        * Host controller initiates
        * Most bus transaction involve transmisison of up to three packets
        * Token packet first
        * Pipes: stream (no defined structure) and message; default control pipe
        * Flow control is offered for stream pipes
        
    * USB offers robustness: error detection and handling
    
    * Attachment goes through a hub, the device receives an address
    
    * Four types of data transfers:
        * Control - configuration and pipe management
        * Bulk - large and bursty, wide latitude and transmission constraints
            * Sequential
            * Reliable
            * Error detection in hardware
            * Variable bandwidth
        * Interrupt - timely and reliable
        * Isochrronous - prenegotiated latency streaming/realtime
    
    * Bandwidth is allocated among pipes; USB devices are required to provide some buffering; of course bandwidth can be spread out.
    
    * USB devices
        * Device classes TODO (Ch 11)
        * All devices support at least one pipe at endpoint zero 9the control pipe)
        
    * Hubs are a thing, but we don't care about them
        * (Check if we are a single hub)
        * Attachment point is a "port"
    
    * Function is an USB device.  
    
    * USB Host interacts with USB devices through the Host Controller.  Its responsibilities are managing the USB devices.
    
    * Blah blah, usb allows for complex topology, but we will only be using host-root hub-device/function.
        
    * Device endpoint
        * Uniquely identifiable portion of an USB device that is the terminus of a communication flow between host and device.
        * Endpoints are independent
        * Endpoint numbers, device-determined direction of data flow
        * <device address, endpoint number, direction> = unique reference
        * Endpoints are simplex with data flow in a single direction.
        * USB Pipe is an association between endpoint on device and software on host
            * Data transfers are requested via I/O Request Packets (IRPs) to a pipe.
            * On STALL or errors for IRPs, IRP is aborted/retried, all outstanding IRPs are retried, and no further IRPs are accepted until software recovers and acknowledges halt/error condition via USBD call
            * Maximum packet size
            * Endpoint can inform that it's busy by responding with NAK.  Not a retry condition.  Is not an error.
            
            * Stream pipes
                * No USB-required structure
                * Unidirectional FIFO
                * Bulk/isochronous/interrupt
            * Message pipes (5.3.2.2) TODO if necessary
        * 5.3.3 frames and microframes
        * transfer types (again)
            * control
            * iso
            * interrupt
            * Bulk transfers
                * Large amounts of data, variable times
                *Access to the USB on a bandwidth-available basis
                * Retry of transfers
                * Guaranteed delivery of data, no guarantee of bandwidth and latency
                * Payload sizes: 8 to 512 (high-speed)
                * Bulk transfers and control registers deliver independently and possibly before each other
                    * Page 53: bulk transfer transaction limits
        * 5.9 high-speed, high bandwidth endpoints?
            * up to 192 Mb/s
        * Transfer management
        * oops I got lost in the depth of this bullet point list
                    
    page 62.
                
    
    
\end{comment}
    \todo{SPEEDS (and average TIMES)}


