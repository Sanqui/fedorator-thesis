\label{Introduction}
    In today's world, there are many free and open–source operating systems based on the Linux kernel, called Linux distributions\cite{whatislinux}.  Despite Linux's relatively small market share (between 2.4\% and 4.6\%, to date\cite{linuxmarketshare}), it is being used by millions of people daily and has significant traction among developers\cite{sosurvey} and technically minded people in general.  The companies and communities behind Linux distributions are, of course, trying to raise the usage of their respetive operating systems.  One way to achieve that is through marketing stands at various technical conferences.  Besides persuasion and marketing items such as stickers, a great way to get people to try a new operating system is to simply hand it over to them.

    The community behind the Linux distribution called Fedora\cite{fedora} has been giving out live DVDs which let people try and install Fedora at home.  However, recently a decision was made and Fedora 25 is the last release which will be distributed on DVDs in the European region.  This is because as time advances, less computers actually have a DVD drive.  In the past few years, laptops have been getting thinner and come without DVD drives\cite{laptopdvd}.  Most computers nowadays are capable of booting from an USB flash drive just as well as from disk drives.
    
    The goal of this thesis is to design and construct a device, dubbed the "Fedorator", which will enable writing Fedora to flash drives as a self-service.  The primary use of the Fedorator would be on marketing stands, where it would be available for visitors to use with any USB flash drive.  This device should be relatively simple to construct using off-the-shelf components.
    \section{How Fedora Spreads Publicity}
        The Fedora project has a worldwide Ambassador program.  Ambassadors are people who act as representatives of Fedora.  They have a good understanding of Fedora's principles and aim to spread the message to the public\cite{fedora-ambassadors}.  One of the tasks assigned to Fedora Ambassadors is organizing Fedora participation at events.
        \todo{Discusson over Fedora stands on events and conferences}
        \todo{How DVDs are in use}
        \todo{Ask some people involved in advertisement?}
        \blind[3]
    \section{Rationale for a New Solution}
        Why design and build the Fedorator?  Is there not some existing solution that would prove sufficient?
        
        \subsection{Preloaded USB flash drives}
            Many services offering bulk preloaded USB flash drives may be found on the Internet.  With a large focus on marketing, in general they offer to preload materials such as PowerPoint presentations and PDF prochures\cite{flashbay-data-preloading}.  Literally, they instruct to send the files by email or upload them.  Hence, I'm not sure whether they would be capable of producing bootable USB flash drives.
            
            I went looking to order with the following parameters.
            \begin{itemize}
                \item Capacity of 2GB
                \item 2GB of preloaded data
                \item USB flash drive must be bootable
                \item No branding necessary
                \item 500 units
            \end{itemize}
            
            Most services, rather than showing a cost upfront, ask the user to fill in a form and get a personalized quote.
            
            \todo{ https://www.memorysuppliers.com/ http://www.flashbay.com https://www.premiumusb.com/usb-preloading http://www.memotrek.com/usb/usb-flash-drives-clip-n-easy : 4.19 USD}
            
        \todo{Discuss how USB flash drives could be mass produced here, and why it isn't quite desirable.}
        \todo{Also mention existing USB flash drive cloning devices and how expensive they are.}
        \todo{How much can I simply lift this from the positional report?}
        \blind[3]
