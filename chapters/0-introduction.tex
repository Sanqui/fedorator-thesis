\label{Introduction}
    In today's world, there are many free and open–source operating systems based on the Linux kernel, called Linux distributions\cite{whatislinux}.  Despite Linux's relatively small market share (between 2.4\% and 4.6\%, to date\cite{linuxmarketshare}), it is being used by millions of people daily and has significant traction among developers\cite{sosurvey} and technically minded people in general.  The companies and communities behind Linux distributions are, of course, trying to raise the usage of their respetive operating systems.  One way to achieve that is through marketing stands at various technical conferences and trade shows.  Besides persuasion and marketing items such as stickers, a great way to get people to try a new operating system is to simply hand it over to them.

    The community behind the free Linux distribution called Fedora\cite{fedora} has been giving out live DVDs which let people try and install Fedora at home.  However, recently a decision was made and Fedora 25 is the last release which will be distributed on DVDs in the European region. \todo{footnote}  This is because as time advances, less computers actually have a DVD drive.  In the past few years, laptops have been getting thinner and come without DVD drives\cite{laptopdvd}.  Most computers nowadays are capable of booting from an USB flash drive just as well as from disk drives\cite{fedora-how-to-live-usb}.
    
    The goal of this thesis is to design and construct a device, dubbed the "Fedorator", which will enable writing Fedora to flash drives as a self-service.  The primary use of the Fedorator would be on marketing stands, where it would be available for visitors to use with any USB flash drive.  This device should be relatively simple to construct using off-the-shelf components, even for a person who haven't seen it in person before.
    
    In this thesis, I will firstly go on to consider the problem space of delivering bootable USB flash drives and look for existing solutions.  I will then justify designing a new device to serve the purpose.
