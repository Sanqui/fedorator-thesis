\chapter{Analysis of Possible Solutions}
    \section{Cost Considerations}
        \todo{Point out the cost, how much money might be allotted towards building instances of the device...}
        \blind[2]
    \section{Ease of Assembly}
        \todo{Who will be assembling it?  What will make it easy to assemble?  What are potential obstacles?}
        \blind[2]
    \section{Hardware Options}
        A device such as this needs a computer at its core.  I shall go over a number of potential candidates, introducing them shortly and weighing their features.
        \subsection{Raspberry Pi}
            \todo{Picture}
            The \textbf{Raspberry Pi} is a small, inexpensive, and fully equipped single board computer.  It was originally developed to facilitate better teaching of computer classes by enthusiasts at Cambridge\cite{bbc-15-pound-computer}.  The original £15 computer went on to be a much bigger success than anticipated and by November 2016, the total sales have reached ten million units\cite{rpi-ten-million}.
            
            Today the Raspberry Pi foundation continues to develop new versions of the computer\cite{rpi-products}.  The latest model is \textbf{Raspberry Pi~3 Model B}, released in February 2016.  It boasts the following features\cite{rpi-3}.
            
            \begin{itemize}
                \item 1.2GHz 64-bit quad-core ARMv8 CPU,
                \item 802.11n Wireless LAN,
                \item Bluetooth 4.1 and Bluetooth Low Energy (BLE),
                \item 1GB RAM,
                \item 4 USB ports,
                \item 40 GPIO pins,
                \item HDMI port,
                \item Ethernet port,
                \item Combined 3.5mm audio jack and composite video,
                \item Camera interface (CSI) and Display interface (DSI),
                \item Micro SD card slot,
                \item VideoCore IV 3D graphics core.
            \end{itemize}
            
            We can see that some of the necessary components are already included: four USB ports, a MicroSD slot.  This means there is less assembly involved.  For example, we may avoid the need to solder entirely.
            
            The dimensions of a Raspberry Pi~3 Model B are 85.60 by 56.5~mm (without connectors).  This may prevent it from being used in a more compact design.
            
            Raspberry Pis are widely available and there are local distributors, or resellers in most places around the world\cite{rpi-buying-guide}\cite{rpi-buying-links-by-country}.  In the Czech Republic, a Raspberry Pi 3 Model 3 may be bought for between 1~046 and 1~359~CZK\cite{rpi-rpi3-rpishop}\cite{rpi-rpi3-minidroid}\cite{rpi-rpi3-alza}.
            
            The Raspberry Pi is open hardware with the exception of the Broadcom SoC.  For an operating system, many Linux distributions provide support\cite{rpi-opensource}.  Being a computer running a full operating system, support for acting as a USB host device comes naturally.
            
            Finally, due to the sheer popularity of Raspberry Pi, there are many compatible components and accessories available\cite{rpi-rs-components}\cite{rpi-the-pi-hut-store}, such as specially designated touchscreens\cite{rpi-official-touchscreen}.
            
            These features, along with the factor of familiarity between hobbyists, make the Raspberry Pi a favorable option for the Fedorator.  The disadvantages include the price, which is being raised by a number of components that are not necessary for the Fedorator, as well as potentially the dimensions.
        \subsection{Arduino}
            Looking for another platform for hardware development, one which may encompass a smaller form factor, we may be quickly brought to the popular \textbf{Arduino} brand.
            
            Arduino as a platform stems from the work of Hernando Barragán, who designed the Wiring development platform in 2003 as a Master's thesis.  His goal was to create a low cost and open platform accessible even to non-engineers\todo{Footnote, "I include this because the history is often misinterpreted, ...}\cite{arduino-untold-history}.  Today the Arduino \todo{}
            
            There is a variety of boards available in the Arduino family\cite{arduino-products}.  All of them come equipped with a microprocessor, usually from the AVR family, and a set of analog and digital pins.  Some include an USB interface for communication with a computer.
            
            \todo{arduinos \& their prices, also clones}
            \todo{There may be problems with slow CPU and little memory}
            \todo{Pictures}
            \subsubsection{Uno}
            
            \subsubsection{Mini}
            
        \subsection{ESP8266}
            The \textbf{ESP8266} is a low-cost Wi-Fi chip.  
            
            \todo{nodemcu, faster than arduino but not a good choice}
        \subsection{Teensy}
            \todo{Picture}
            \textbf{Teensy} is a family of USB development boards boasting the ARM microprocessor.  It is created by Paul Stoffregen and the latest versions, Teensy~3.5 and Teensy~3.6, were funded successfully through Kickstarter\cite{teensy-35-36-kickstarter}.
            
            \textbf{Teensy~3.6} comes equipped with the the following.
            
            \begin{itemize}
                \item 180 MHz ARM Cortex-M4 with Floating Point Unit,
                \item 1M Flash, 256K RAM, 4K EEPROM,
                \item USB Full Speed (12 Mbit/sec) port,
                \item USB High Speed (480 Mbit/sec) port,
                \item Native micro SD card port,
                \item CAN bus ports, serial, SPI, I2C, I2S, Ethernet and more.
            \end{itemize}
            
            The size is stated as 2.4 by 0.7~inch (about 6 by 1.8~cm).
            
            Teensy~3.6 is officially available for 29.25~USD (about 715~CZK) without shipping\cite{teensy-36}, or, for example, at local distributors in the Czech Republic for 1~198~CZK\cite{teensy-36-snailshop}.  The cost is comparable with a Raspberry Pi.  \todo{Unlike pi, ports are not exposed?}
            
            The fast CPU, on-board USB host port and MicroSD host port make the Teensy a strong contender.
            
            
        \subsection{FPGA}
            One final option I considered is to forgo a CPU completely and design a custom chip instead.  FPGA stands for field-programmable gate array and it's a class of integrated circuits which can be reconfigured at will.  Thanks to allowing exact configuration of logic elements and memory blocks to serve a specific purpose, they're extremely efficient at signal processing\cite{sadrozinski2016applications}.
            
            By designing custom circuitry to handle copying of data from an SD card to a USB flash drive, the copier could be very fast, bottlenecked only by the connected devices.
            
            Popular FPGA manufacturers include big names such as Xilinx\cite{fpga-xilinx} and Intel\cite{fpga-intel}.  The field of open hardware FPGA tools is lacking, although there do seem to be some contenders, like Papilio\cite{fpga-papilio}. 
            
            However, programming FPGAs is a task more difficult than software programming, requiring the management of input and output signals and careful timing.  I have deemed the task of implementing the USB protocol at such a low-level complex.
            
            
        \todo{Batteries are not an option}
        \blind[4]
    \section{Component Options}
        \todo{Where can buttons and LCDs come from..?}
    \section{Casing Options}
        When it comes to casing, there aren't many options present.
        \todo{More options}
        \subsection{Wood}
            A wooden build would certainly look interesting, even if not modern.  However, because it would have to be handmade, it's not a practical choice.  Few people have wood crafting skills and would invest time in building the moderately complex case.
        \subsection{Plastic}
            Generally, plastic casings require ordering a mold, which is an expensive proposition unless an option unless a significant amount of devices is manufactured.
        \subsection{3D printed}
            In tech circles, 3D printers a common find these days.  It is likely anybody wishing to build a Fedorator would have access to one.  A 3D printed case is simply the matter of getting ahold of a printer.
            
            There may be inconsistencies in color.  \todo{?  Sturdiness?}
            
            For a technically hobby project like this, a 3D printed case is the logical choice.
    \section{Roundup}
        \todo{Include a table and short discussion over final choice(s)}
        \blind[2]
    \section{Concepts}
        Based on this analysis, I have created three distinct Fedorator concepts.
        \pagebreak
        \subsection{Concept A}
            \svgfigure{concept-a}{concept-a}{Concept A}
            The first concept is a moderately large device which is to be fixed in place and used from a standing posture.  It provides an LCD with on-screen information on the status and progress.  The rotary knob allows for controling the menu and making a selection.  It may be replaced with three buttons if so desired.  There are four USB slots present, each with a corresponding status light.  In order to be stable, the device needs support from the back.
            
            Due to the size of the display and the number of USB ports, putting a Raspberry Pi computer at the core makes the most sense.
            
            The Fedorator needs to be plugged into a power source, so there will be a cable running from the back.
            \newpage
        
        \subsection{Concept B}
            \svgfigure{concept-b}{concept-b}{Concept B}
            The second concept is similar in some ways to the first one, but with some tweaks.  The knob is removed, the sole control is a prominent touchscreen positioned vertically.  As a consequence, the device can be thinner.  There are now only two USB slots.  The device is sleek and symmetric.
            
            In order not to have stability issues, it may be necessary to ensure the Fedorator is heavy enough in order to keep the Fedorator in place as it's being used.
            
            This design is also powered by a Raspberry Pi.
            \newpage
        
        \subsection{Concept C}
            \svgfigure{concept-c}{concept-c}{Concept C}
            The third and final concept is a radical departure from the first two.  Instead of standing still while being manipulated, this Fedorator is designed to be picked up and held in a hand.  The OLED screen is small yet sharp and despite providing only a binary image it works well enough for our purposes.  The three buttons allow for selecting the desired live image.
            
            Due to its small size, this device needs a small system at its core.  In an attempt to keep the costs low in contrast to other designs, I will attempt to use the Arduino platform.
            
            This design still needs to be powered by cable.  This cable will be permanently attached by the means of enclosure.  This has the positive side effect of discouraging people from attempting to leave with the Fedorator.
            
            Because it only has a single USB slot, it may be desirable to provide multiple instances of this Fedorator.
            \newpage
            
    \section{Decision}
        I have showed five people these concepts and asked them for their feedack.  In general, response to concept A was lukewarm, however, people were excited about both concept B and C.  Because they differ significantly, I will strive to construct a prototype for each.

